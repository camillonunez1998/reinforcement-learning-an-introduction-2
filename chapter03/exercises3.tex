%! Author = melek
%! Date = 9.06.2022

% Preamble
\documentclass[11pt]{article}

% Packages
\usepackage{amsmath}
\DeclareMathOperator*{\argmax}{argmax}

%%%Mis paquetes
\usepackage{xcolor}
\def\red{\color{red}}
\usepackage{amsfonts}

%Title
\title{RL Excercise Chapter 3}

% Document
\begin{document}

    \maketitle
    \setcounter{section}{2}


    \section{Exercises}

    \subsection{Question}
    Devise three example tasks of your own that fit into the MDP framework,
    identifying for each its states, actions, and rewards. Make the three examples as different from each other as possible. The framework is abstract and flexible and can be applied in
    many different ways. Stretch its limits in some way in at least one of your examples

    \subsection*{Answer}

    An example of this is the topic of my internship, so I will just leave this one for later.

    \subsection{Question}
    Is the MDP framework adequate to usefully represent all goal-directed learning tasks?
    Can you think of any clear exceptions?

    %%%Este ejercicio realmente no lo revisé.

    \subsection*{Answer}
    The main thing about the MDP is that Markov property. There are tasks where this does not hold. For instance, in Poker, the previous states will determine what is in the deck and what is not. Since, in general, states do not store all the relevant information from previous states in this game, it does not obey Markov property.

    \subsection{Question}
    Consider the problem of driving.
    You could define the actions in terms of the accelerator, steering wheel, and brake, that is, where your body meets the machine.
    Or you could define them farther out—say, where the rubber meets the road, considering your actions to be tire torques.
    Or you could define them farther in—say, where your brain meets your body, the actions being muscle twitches to control your limbs.
    Or you could go to a really high level and say that your actions are your choices of where to drive.
    What is the right level, the right place to draw the line between agent and environment?
    On what basis is one location of the line to be preferred over another?
    Is there any fundamental reason for preferring one location over another, or is it a free choice?

    \subsection*{Answer}
    It depends on the problem we are trying to solve.

    Car driving is designed to be a function of acceleration, steering wheel and the brake.
    This interface is already present in any car.
    Natural approach would be to exploit that interface.

    Now let's consider using a lower level control and use brain signals.
    In order to drive the brain process and produces many signals.
    Much of the processed signals are even may not always be related to the driving.
    Produced signals are do not directly control the car but the body.
    More work is needed to translate the signals.
    The brain level control introduces a lot of extra work.

    Now let's move to a higher level and use choices of locations as actions.
    In this case it will be necessary to relate all the locations, road conditions e.g. pedestrians, lights, and car state e.g. turning, acceleration which produces an infinite MDP.
    It is very hard to learn in this setting.

    \subsection{Question}

    Give a table analogous to that in Example 3.3, but for p(s', r |s, a).
    It should have columns for s, a, s 0 , r, and p(s', r| s, a), and a row for every 4-tuple for which p(s 0 , r |s, a) > 0

    \subsection*{Answer}

    The events in which we suppose that $S_t$ and $A_t$ have already fixed values, also happens to have the property that $\mathbb{E}[R_t+1]$ is constant it all the subevents of the form $S_{t+1}=s'$. This basically means that computing $p(s',r|s,a)$ is reduced to computing $p(s'|s,a)$.

    \begin{table}[h!]
        \centering
        \begin{tabular}{ll|ll|c}
            $s$ & $a$ & $s'$ & $r$ & $p(s', r | s, a)$ \\
            \hline
             \texttt{high}& \texttt{search}&  \texttt{high}&  $r_{\texttt{search}}$& $\alpha$  \\
             \texttt{high}& \texttt{search}&  \texttt{low}&   $r_{\texttt{search}}$& $1 - \alpha$ \\
             \texttt{high}& \texttt{wait}&  \texttt{high}&   $r_{\texttt{wait}}$&  $1$ \\
             \texttt{low}& \texttt{recharge}&   \texttt{high}&    $0$&   $1$\\
             \texttt{low}& \texttt{search}&  \texttt{high}&  $-3$& $1 - \beta$  \\
             \texttt{low}& \texttt{search}&  \texttt{low}&   $r_{\texttt{search}}$& $\beta $\\
             \texttt{low}& \texttt{wait}&  \texttt{low}&   $r_{\texttt{wait}}$&  $1$
        \end{tabular}
        
        
        \caption{Transition table}
        %\label{table:3.4}
        \end{table}
        
    \subsection{Question}

    The equations in Section 3.1 are for the continuing case and need to be modified (very slightly) to apply to episodic tasks.
    Show that you know the modifications needed by giving the modified version of (3.3).

    \subsection*{Answer}

    \begin{equation}
        \sum_{s' \in \mathcal{S}^+}^{} \sum_{r \in \mathcal{R}}^{} p(s', r | s, a) = 1 \text{ for all } s \in \mathcal{S}, a \in \mathcal{A}(s), \mathcal{S}^+ \text{being all states, } \mathcal{S} \text{ being non-terminal states}.
    \end{equation}

    \subsection{Question}

    Suppose you treated pole-balancing as an episodic task but also used discounting, with all rewards zero except for 1 upon failure.
    What then would the return be at each time?
    How does this return differ from that in the discounted, continuing formulation of this task?

    \subsection*{Answer}

    In this case, Episodic + Discounting, the return at each time would be:

    \begin{equation}
        G_t = - \gamma^{T - t}.
    \end{equation}

    In the continuing formulation it would be:

    \begin{equation}
        G_t = - \sum_{k \in \mathcal{K}} \gamma^{k - t},
    \end{equation}
    where $\mathcal{K}$ is the set of times after $t$ at which the pole falls over.

    Both formulations should work correctly.



    \subsection{Question}

    Imagine that you are designing a robot to run a maze.
    You decide to give it a reward of +1 for escaping from the maze and a reward of zero at all other times.
    The task seems to break down naturally into episodes—the successive runs through the maze—so you decide to treat it as an episodic task, where the goal is to maximize expected total reward (3.7).
    After running the learning agent for a while, you find that it is showing no improvement in escaping from the maze.
    What is going wrong?
    Have you effectively communicated to the agent what you want it to achieve?

    \subsection*{Answer}

    If you do not use discounting you are assigning the same return value to a policy taking 100 steps to find the way out than to a policy taking only 5 steps. That is the problem.

    \subsection{Question}

    Suppose $\gamma=0.5$0 and the following sequence of rewards is received R1 = -1, R2 = 2, R3 = 6, R4 = 3, and R5 = 2, with T = 5.
    What are G0, G1,\ldots, G5 ?
    Hint: Work backwards.

    \subsection*{Answer}

    \begin{equation}
        G_t = R_{t+1} + \gamma G_{t+1}
    \end{equation}

    $G_5 = 0$

    $G_4 = R_{5} + \gamma G_{5} = 2 + 0.5 * 0 = 2 $

    $G_3 = R_{4} + \gamma G_{4} = 3 + 0.5 * 2 = 4 $

    $G_2 = R_{3} + \gamma G_{3} = 6 + 0.5 * 4 = 8 $

    $G_1 = R_{2} + \gamma G_{2} = 2 + 0.5 * 8 = 6 $

    $G_0 = R_{1} + \gamma G_{1} = -1 + 0.5 * 6 = 2 $

    \subsection{Question}

    Suppose $\gamma= 0.9$ and the reward sequence is R1 = 2 followed by an infinite sequence of 7s.
    What are G1 and G0 ?

    \subsection*{Answer}

    \begin{equation}
        G_t = R_{t+1} + \gamma G_{t+1}
    \end{equation}

    \begin{equation}
        G_t = \sum_{k=0}^{\in} \gamma ^ k = \frac{1}{1-\gamma} \text{ given } \gamma < 1
    \end{equation}

    $G_1 = 7 * \frac{1}{1-0.9} = 70 $

    $G_0 = R_{1} + \gamma G_{1} = 2 + 0.9 * 70 = 65 $

    \subsection{Question}

    Prove the second equality in (3.10).

    \subsection*{Answer}

Supposing we know the series converges,

$\gamma + \gamma^2 + \gamma^3 + \dots = k$

$\Leftrightarrow \gamma(1+\gamma + \gamma^2 + \dots) = k$ 

$\Leftrightarrow \frac{k}{\gamma} = k + 1 \Leftrightarrow k = k\gamma + \gamma$

$\Leftrightarrow k = \frac{\gamma}{1-\gamma}$.

Alternatively, equation $\sum_{k=0}^{\infty}a*r^k$ is sum of geometric series which equals to $\frac{a}{1-r}$ if $r<1$.

In equation (3.10) $a=1$ and $r = \gamma = 0.9$

    \subsection{Question}

    If the current state is St, and actions are selected according to stochastic policy pi, then what is the expectation of R t+1 in terms of pi and the four-argument function p(3.2)?

    \subsection*{Answer}

    \begin{equation}
        E_{\pi}[R_{t+1}|S_{t}=s] = \sum_{a}^{} \pi(a|s) \sum_{s', r}^{} p(s',r|s, a) * r
    \end{equation}

    \subsection{Question}

    Give an equation for $V_{\pi}$ in terms of $q_\pi$ and $\pi$.

    \subsection*{Answer}

    \begin{equation}
        v_{\pi}(s) = \sum_{a \in A(s)}^{} \pi(a|s)q_{\pi}(a,s)
    \end{equation}

    \subsection{Question}

    Give an equation for $q_{pi}$ in terms of $v_{pi}$ and the four-argument p.

    \subsection*{Answer}

    \begin{equation}
        q_{\pi}(s, a) = \sum_{s' \in \mathcal{S}}^{} v_{\pi}(s')*p(s', r | s, a)
    \end{equation}

    Where $r = \mathbb{E}[R_{t+1}|S_t = s, A_t = a]$.

    \subsection{Question}

    The Bellman equation (3.14) must hold for each state for the value function $v_{pi}$ shown in Figure 3.2 (right) of Example 3.5.
    Show numerically that this equation holds for the center state, valued at +0.7, with respect to its four neighboring states, valued at +2.3, +0.4, 0.4, and +0.7.
    (These numbers are accurate only to one decimal place.)

    \subsection*{Answer}

    \begin{equation}
        v_{\pi}(s_{center}) = 0.25 * 0.9 ( 2.7 + 0.3 + 0.4 -0.4 ) = 0.675 \approx 0.7
    \end{equation}

    \subsection{Question}

    In the gridworld example, rewards are positive for goals, negative for running into the edge of the world, and zero the rest of the time.
    Are the signs of these rewards important, or only the intervals between them?
    Prove, using (3.8), that adding a constant c to all the rewards adds a constant, v c , to the values of all states, and thus does not affect the relative values of any states under any policies.
    What is v c in terms of c and ?

    \subsection*{Answer}

    \begin{equation}
        G_{t} = \sum_{k=0}^{\infty} \gamma^{k} R_{t+k+1}
    \end{equation}

    Adding constant c:

    \begin{equation}
        G'_{t} = \sum_{k=0}^{\infty} \gamma^{k} (R_{t+k+1} + c) = \sum_{k=0}^{\infty} \gamma^{k} R_{t+k+1} + \gamma^{k} c = G_{t} +  \sum_{k=0}^{\infty} \gamma^{k} c
    \end{equation}

    \begin{equation}
        G'_{t} =  G_{t} + \frac{c}{1-\gamma}
    \end{equation}

    Showing $v'$ in terms of c and $\gamma$

    \begin{equation}
        v'(s) =  E[G'_{t}|S_{t}=s] = E[G_{t} + \frac{c}{1-\gamma}|S_{t}=s] = E[G_{t}|S_{t}=s] + \frac{c}{1-\gamma} = v(s) + \frac{c}{1-\gamma}
    \end{equation}

    \subsection{Question}

    Now consider adding a constant c to all the rewards in an episodic task, such as maze running.
    Would this have any effect, or would it leave the task unchanged as in the continuing task above?
    Why or why not?
    Give an example.

    \subsection*{Answer}

    In an episodic task adding a constant may change the behaviour.
    If c is big enough it may assign a positive reward to taking time in the maze which makes the agent prefer not to exit the maze.

    \subsection{Question}

    What is the Bellman equation for action values, that is, for $q_\pi$ ?
    It must give the action value $q\pi(s, a)$ in terms of the action values, $q\pi(s', a')$, of possible successors to the state–action pair (s, a).
    Hint: the backup diagram to the right corresponds to this equation.
    Show the sequence of equations analogous to (3.14), but for action values.

    \subsection*{Answer}

    $q(s, a) =  E_{\pi}[G_{t}|S_{t}=s, A_{t}=a] = E_{\pi}[R_{t+1} + \gamma G_{t+1}|S_{t}=s, A_{t}=a]$

    $q(s, a) =  \sum_{s',r} p(s',r | s, a) [r + \gamma \sum_{a' \in A(s')} \pi(a'|s')q(s', a') ] $

    \subsection{Question}

    The value of a state depends on the values of the actions possible in that state and on how likely each action is to be taken under the current policy.
    We can think of this in terms of a small backup diagram rooted at the state and considering each possible action.
    Give the equation corresponding to this intuition and diagram for the value at the root node, v pi (s), in terms of the value at the expected leaf node, q pi (s, a), given St = s.

    \subsection*{Answer}

    $v_{\pi}(s) = E_{\pi}[G_{t}|S_{t}=s] = E_{\pi}[q_{\pi}(s,a)|S_{t}=s, a \in A(s)] $

    $v_{\pi}(s) = \sum_{a \in A(s)} \pi(s, a) q_{\pi}(s,a) $

    \subsection{Question}

    The value of an action, q pi (s, a), depends on the expected next reward and the expected sum of the remaining rewards.
    Again we can think of this in terms of a small backup diagram, this one rooted at an action (state–action pair) and branching to the possible next states.
    Give the equation corresponding to this intuition and diagram for the action value, q pi (s, a), in terms of the expected next reward, R t+1 , and the expected next state value, v pi (S t+1 ), given that S t = s and A t = a.

    \subsection*{Answer}

    $ q_{\pi}(s, a) = E[G_{t}|S_{t}=s, A_{t}=a] = E[R_{t+1} + \gamma v_{\pi}(s') |S_{t}=s, A_{t}=a, s' \in A(s)] $

    $ q_{\pi}(s, a) = \sum_{s',r} p(s', r | s, a) [r + \gamma v_{\pi}(s')] $

    \subsection{Question}

    Draw or describe the optimal state-value function for the golf example.

    \subsection*{Answer}

    It is optimal to use putter inside the green region, elsewhere driver is the optimal club.

    State value for the green region is -1;
    State value for the contour where green region can be reached in one shot using a driver is -2;
    State value for the contour where region with state value -2 can be reached in one shot using a driver is -3;

    \subsection{Question}

    Draw or describe the contours of the optimal action-value function for putting, q*(s, putter), for the golf example.

    \subsection*{Answer}

    Action value for the putter inside the green region is -1;
    Action value for the putter for the contour where green region can be reached in one shot using a putter is -2;
    Action value for the putter for the outer contour 1 is -3 because after first putter two more shots are needed in any case.
    Action value for the putter for the outer contour 2 is -3 because after first putter using one driver and one putter may be optimal.
    As we go out regions will be larger than first two regions.
    Overall, since using putter outside green region is sub-optimal, action values will be lower compared to q*(s, driver) illustrated in the book.

    \subsection{Question}

    Consider the continuing MDP shown on to the right.
    The only decision to be made is that in the top state, where two actions are available, left and right.
    The numbers show the rewards that are received deterministically after each action.
    There are exactly two deterministic policies, policy left and policy right .
    What policy is optimal if $\gamma = 0$? If = $\gamma = 0.9$? If $\gamma = 0.5$?

    \subsection*{Answer}

    $ v_{\pi}(s_{0}) = E_{\pi}[G_{0}] =  E_{\pi}[R_{1} + \gamma G_{1}] $

    $ v_{\pi}(s_{top}) = \sum_{k=0} \gamma^{2^{k}}a + \sum_{k=0} \gamma^{2^{k}}\gamma*b $

    $ v_{\pi}(s_{top}) = \frac{a}{1-\gamma^{2}} + \frac{\gamma * b}{1-\gamma^{2}} $

    $ \text{let } d = \frac{1}{1-\gamma^{2}} $

    $ v_{left}(s_{top}) = \frac{1}{1-\gamma^{2}} + \frac{\gamma * 0}{1-\gamma^{2}} = \frac{1}{1-\gamma^{2}} = d $

    $ v_{right}(s_{top}) = \frac{0}{1-\gamma^{2}} + \frac{\gamma * 2}{1-\gamma^{2}} = \frac{\gamma * 2}{1-\gamma^{2}} = 2 \gamma d $

    \subsubsection*{Case $\gamma=0$}

    $ v_{left}(s_{top} | \gamma=0) = d $
    $ v_{right}(s_{top} | \gamma=0) = 0 $

    Optimal action is left.


    \subsubsection*{Case $\gamma=0.5$}

    $ v_{left}(s_{top} | \gamma=0.5) = d $
    $ v_{right}(s_{top} | \gamma=0.5) = d $

    Optimal action is left or right.

    \subsubsection*{Case $\gamma=0.9$}

    $ v_{left}(s_{top} | \gamma=0.9) = d $
    $ v_{right}(s_{top} | \gamma=0.9) = 1.8d =  $

    Optimal action is right.

    \subsection{Question}

    Give the Bellman equation for q* for the recycling robot.

    \subsection*{Answer}

    $ q_{*}(s, a) = E[R + \max_{a'} q(s',a') ] $

    $ q(h, s) = \alpha [r_{search} + \max_{a'} q(h,a')] + (1-\alpha) [r_{search} + \max_{a'} q(l,a') ]  $

    $ q(l, s) = \beta [r_{search} + \max_{a'} q(l,a')] + (1-\beta) [-3 + \max_{a'} q(h,a') ]  $

    $ q(h, w) = r_{wait} + \max_{a'} q(h,a') $

    $ q(l, w) = r_{wait} + \max_{a'} q(l,a') $

    $ q(l, w) = \max_{a'} q(h,a') $

    \subsection{Question}

    Figure 3.5 gives the optimal value of the best state of the gridworld as 24.4, to one decimal place.
    Use your knowledge of the optimal policy and (3.8) to express this value symbolically, and then to compute it to three decimal places.

    \subsection*{Answer}

    $G_{t} = \sum_{k=0} \gamma ^ {k} R_{t+k+1} $

    Optimal policy is to go from A' to A.

    $V_{*}(A) = 10 + \gamma*0 + \gamma^{2}*0 + \gamma^{3}*0 + \gamma^{4}*0 + \dots $

    $V_{*}(A) = \sum_{k=0} \gamma^{5k} (10 + \gamma*0 + \gamma^{2}*0 + \gamma^{3}*0 + \gamma^{4}*0) = \sum_{k=0} \gamma^{5k} 10 = \frac{10}{1-\gamma^5}$

    $V_{*}(A) = \frac{10}{1-0.9^5} = 24.419 $

    \subsection{Question}

    Give an equation for v* in terms of q*.

    \subsection*{Answer}

    $ v_{*}(s) = \max_{a} q_{*}(s,a) $

    \subsection{Question}

    Give an equation for q* in terms of v* and the four-argument p.

    \subsection*{Answer}

    $ q_{*}(s, a) = \sum_{s', r} p(s', r | s, a) ( r + \gamma v*(s') ) $

    \subsection{Question}

    Give an equation for $ \pi_{*} $ in terms of $ q_* $ .

    \subsection*{Answer}

    \begin{equation}
        \pi_*(a|s) =
        \begin{cases}
            1,& \text{ if } a = \argmax_{a'}q_{*}(a'|s) \\
            0,& \text{ else }
        \end{cases}
    \end{equation}

    \subsection{Question}

    Give an equation for pi* in terms of v* and the four-argument p.

    \subsection*{Answer}

    \begin{equation}
        \pi_*(a|s) =
        \begin{cases}
            1,& \text{ if } a = \argmax_{a'} \sum_{r,s'} p(r, s'| a', s)(r + \gamma v_*(s')) \\
            0,& \text{ else }
        \end{cases}
    \end{equation}

    \subsection{Question}

    Rewrite the four Bellman equations for the four value functions $ (v_\pi ,v_* , q_\pi \text{ and } q_* ) $ in terms of the three argument function p (3.4) and the two-argument function r (3.5).

    \subsection*{Answer}

    \begin{equation}
        G_t = R_{t+1} + \gamma G_{t+1}
    \end{equation}

    \begin{equation}
        p(s'|s,a) = Pr\{S_t=s'|S_{t-1}=s, A_{t-1}=a\} = \sum_{r \in R} p(s', r | s, a)
    \end{equation}

    \begin{equation}
        r(s,a) = E [R_t |S_{t-1}=s, A_{t-1}=a ] = \sum_{r \in R, s' \in S} p(s', r | s, a) * r
    \end{equation}

    \begin{equation}
        v_*(s) = \max_{a} \sum_{s',r} p(s',r| s, a) [r+\gamma v_*(s')]
    \end{equation}

     \begin{equation}
        q_*(s, a) = \sum_{s',r} p(s',r| s, a) [r+\gamma \max_{a'} q_*(s',a')]
    \end{equation}

    \subsubsection*{Derivations}

    \begin{equation}
        \begin{split}
            v_{\pi} (s) = E_{\pi} [G_t | S_t = s] = E_{\pi} [ R_{t+1} + \gamma G_{t+1} | S_t = s] \\
            v_{\pi} (s) = \sum_{a} \pi(a|s) \sum_{s',r} p(s',r|s,a) [r+\gamma v_\pi(s')] \\
            v_{\pi} (s) = \sum_{a} \pi(a|s) [\sum_{s',r} p(s',r|s,a) r + \sum_{s'}\sum_{r} p(s',r|s,a) \gamma v_\pi(s')] \\
            v_{\pi} (s) = \sum_{a} \pi(a|s) [r(s,a) + \sum_{s'} p(s'|s,a) \gamma v_\pi(s')] \\
        \end{split}
    \end{equation}

    \begin{equation}
        \begin{split}
            q_{\pi} (s, a) = E_{\pi} [G_t | S_t = s, A_t = a] = E_{\pi} [ R_{t+1} + \gamma G_{t+1} | S_t = s , A_t = a] \\
            q_{\pi} (s, a) = \sum_{s',r} p(s',r|s,a) [r+\gamma v_\pi(s')] \\
            q_{\pi} (s, a) = \sum_{s',r} p(s',r|s,a) r + \sum_{s'}\sum_{r} p(s',r|s,a) \gamma v_\pi(s') \\
            q_{\pi} (s, a) = \sum_{a} \pi(a|s) [r(s,a) + \sum_{s'} p(s'|s,a) \gamma v_\pi(s') \\
        \end{split}
    \end{equation}

    \begin{equation}
        \begin{split}
            v_* (s) = \max_{a} [\sum_{s',r} p(s',r| s, a) r+ \sum_{s'} \sum_{r} p(s',r| s, a) \gamma v_*(s')] \\
            v_* (s) = \max_{a} [ r(s,a) +  \sum_{s'} p(s'| s, a) \gamma v_*(s') ]
        \end{split}
    \end{equation}

    \begin{equation}
        \begin{split}
            q_* (s, a) =  \sum_{s',r} p(s',r| s, a) * r+ \sum_{s'} \sum_{r} p(s',r| s, a) * \gamma \max_{a'} q_*(s',a') \\
            v_* (s) = r(s, a) + \sum_{s'} p(s'| s, a) * \gamma \max_{a'} q_*(s',a')
        \end{split}
    \end{equation}

\end{document}